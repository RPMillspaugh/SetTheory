
\chapter*{}

\vfill
\noindent
\begin{tabular}{ m{1.8in} m{3.7in}}
\includegraphics{by-nc-sa.eps}  & \copyright 2018 Richard Millspaugh \hfill\break This work is licensed under the Creative Commons Attribution-NonCommercial-ShareAlike 4.0 International License. To view a copy of this license, visit http://creativecommons.org/licenses/by-nc-sa/4.0/. \\
\end{tabular} 
\vfill

\chapter*{Preface}

This text is appropriate for a transition to abstract mathematics course that covers basic set theory, an introduction to the real numbers, and some cardinality. It grew from my notes for such a course at the University of North Dakota, which is usually taken by math majors during their sophomore year. Except for a few motivational examples in the early chapters, the text is as self-contained as possible and does not assume much prerequisite material, though it is helpful to have some mathematical maturity before attempting to read the text. In Chapters 6 and 7, I assume the reader is familiar with the material in Chapters 1 -- 5, but Chapters 6 and 7 are independent of each other. The style throughout is informal. 

In using this text for my course, I always cover the material in Chapters 1 -- 5 in depth. At various times I have finished the semester with either Chapter 6, Chapter 7, or both if time permits. I have included a proof of the Cantor-Bernstein Theorem in the appendix for completeness, but the proof is beyond the reach of many students who are just beginning to read and write proofs on their own.

\clearpage
